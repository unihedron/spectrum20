{\Large {\bfseries E}lectro{\bfseries m}agnetic {\bfseries R}adiation (EMR)}
\begin{itemize}
\item EMR is emitted in discrete units called photons but has properties of waves. EMR can be created by the oscillation or acceleration of electrical charge or magnetic field. EMR travels through space at the speed of light ($c=2.997 924 58 \times 10^{8}$ \si{\metre\per\second} ). EMR consists of an oscillating electrical and magnetic field at right angles to each other and spaced at a particular wavelength.

\rput[t]{0}(0.7,-.5){%Description: Electromagnetic radiation schematic with travelling photon

\definecolor{DarkGreen}{rgb}{0,0.6,0}
\definecolor{MagneticBlue}{rgb}{.7,.7,1}
\definecolor{ElectricRed}{rgb}{1,.7,.7}

\psset{unit=0.5in}%Size

\psset{viewpoint=1 1 1}

\psset{hatchsep=2pt}

%Axis labels
\ThreeDput[normal=1 0 0 ]{\textcolor{ElectricRed}{
	\psline[linecolor=ElectricRed]{<->}(0,-1.1)(0,1.1)
%	\uput{2pt}[90](0,1.5){+Electric Field Strength}
	\rput[l]{0}(-0.1,1.3){Electric Field Strength}
	\uput{2pt}[270](0.3,-0.8){-E}}
}

\ThreeDput[normal=0 0 1,embedangle=90]{\textcolor{MagneticBlue}{
	\psline[linecolor=MagneticBlue]{<->}(0,-1.1)(0,1.1)
	\rput[l]{0}(-0.1,-1.3){Magnetic Field Strength}
	\uput{2pt}[90](0.3,0.8){-B}}
}

% Source and Space labels
\ThreeDput[normal=1 0 0]{
       \psline[linecolor=white,linestyle=solid]{cc->}(0,0)(3.4,0)
       \uput{5pt}[180](0,0.1){\white Source}
       \uput{4pt}[270](3.2,0){\white Space}
}

{ % Waveforms
	\psset{linestyle=none, fillstyle=hlines,hatchangle=90}

	% Magnetic (back humps)
	\ThreeDput[normal=0 0 1,embedangle=90]{
		\psset{hatchcolor=MagneticBlue}
		\parabola(0,0)(.5,-1)
		\parabola(1,0)(1.5,1)
	}
	% Electric (back humps)
	\ThreeDput[normal=1 0 0]{
		\psset{hatchcolor=ElectricRed}
		\parabola(0,0)(.5,1)
		\parabola(1,0)(1.5,-1)
	}
	% Magnetic (front hump)
	\ThreeDput[normal=0 0 1,embedangle=90]{
		\psset{hatchcolor=MagneticBlue}
		\parabola(2,0)(2.5,-1)
	}
	% Electric (front hump)
	\ThreeDput[normal=1 0 0]{
		\psset{hatchcolor=ElectricRed}
		\parabola(2,0)(2.5,1)
	}
}

\rput[l](0,-1.8){
	\parbox[t]{2in}{
		\textcolor{white}{Wave Nature}
	}
}

}
\rput[t]{0}(2.85,-.5){%Description: Electromagnetic radiation schematic with travelling photon

\definecolor{DarkGreen}{rgb}{0,0.6,0}
\definecolor{MagneticBlue}{rgb}{.7,.7,1}
\definecolor{ElectricRed}{rgb}{1,.7,.7}

\psset{unit=0.5in}%Size

\psset{viewpoint=1 1 1}

\psset{hatchsep=2pt}

%Axis labels
\ThreeDput[normal=1 0 0 ]{
	\psline[linecolor=ElectricRed]{<->}(0,-1.1)(0,1.1)
}

\ThreeDput[normal=0 0 1,embedangle=90]{
	\psline[linecolor=MagneticBlue]{<->}(0,-1.1)(0,1.1)
}


% Source and Space label
\ThreeDput[normal=1 0 0]{
       \psline[linecolor=white,linestyle=solid]{cc->}(0,0)(1.65,0)
       \uput{5pt}[180](0,0.1){\white Source}
       \uput{4pt}[270](1,0){\white Space}
       }

%Draw photon (position approximate since THREED system is confusing and circle must not be squashed)
\rput(1.3,-.74){
	\pscircle[fillstyle=solid,fillcolor=DarkGreen,linecolor=DarkGreen](0,0){0.1}
	\psarc[linestyle=solid,linecolor=white,linewidth=0.5pt]{cc-cc}(0,0){0.065}{100}{150}
	}

\rput[l](-.7,-1.8){
	\parbox[t]{2in}{
		\textcolor{white}{Particle Nature}
	}
}

}

\vspace{1.45in}

\item The particle nature is exhibited when a dimly lit solar cell emits individual electrons. The wave nature is demonstrated by the double slit experiment that shows cancellation and addition of waves.

% \item Much of the EMR properties are based on theories since we can only see the effects of EMR and not the actual photon or wave itself.

% \item Einstein theorized that the speed of light is the fastest that anything can travel. He has not been proven wrong.

\item The wavelength of EMR is changed when the source is receding or approaching. Red-shift makes high-speed receding stars and galaxies appear more red.

% \item We only have full electronic control over frequencies in the microwave range and lower. Higher frequencies must be created by waiting for the energy to be released from elements as photons. We can either pump energy into the elements (ex. heating a rock with visible EMR and letting it release infrared EMR) or let it naturally escape (ex. uranium decay).
% 
% \item We can only see the visible spectrum. All other bands of the spectrum are depicted as hatched colors \psframebox[fillstyle=none,linestyle=none,framesep=0in]{\psframe[linearc=0,framearc=0,fillstyle=crosshatch,linewidth=0pt,linestyle=none, hatchwidth=2pt, hatchsep=1.5pt,hatchcolor=white](0,-.04)(.4,.1)}\hspace{0.4in}.


\end{itemize}
