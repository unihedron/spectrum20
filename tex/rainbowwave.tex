%Rainbowwave.tex

%This feature uses a lot of LaTeX memory.
% As root:
%  edit the texmf.cnf 
%    and change main_memory = 12,435,456
% then run:
%   fmtutil-sys --all

% Description: Rainbow colored sine wave

% http://www.fon.hum.uva.nl/praat/manual/Script_for_creating_a_frequency_sweep.html
%formula M*sin(A*x+B*(x^2)/C)
%A=Start frequency
%B=End frequency-Start frequency
%C=number of samples
%M=multiplying height factor since we need smaller pgfplots height than allowed.
\newcommand{\nRainbowStartFreq}{5}% Start frequency.
\newcommand{\nB}{40}% End frequency - start frequency.
\newcommand{\nC}{1e3}% Number of samples. (was 1e3)
\newcommand{\nM}{0.35}% Multiplying factor for amplitude.

\pgfplotsset{
	compat=1.14,
	axis line style={draw=none},
	tick style={draw=none},
	xticklabels=\empty,
	yticklabels=\empty,
	colormap={redblue}{
		rgb255(0cm)=(128,0,0); rgb255(1cm)=(255,0,0); rgb255(2cm)=(255,255,0);
		rgb255(3cm)=(100,255,0);rgb255(4cm)=(0,255,255); rgb255(5cm)=(0,0,180)},
	samples=8000,%was 1000, but capacity exceeded , looked into -shell escape external method make nonoverlapping pdfs (on separate pages).
	domain=1:\nC%range
}

\begin{tikzpicture}[rotate=90]
	\begin{axis}[width=48in,height=0.9in]
		\addplot [white,line width=3pt,line join=round, line cap=round]             {\nM * sin(\nRainbowStartFreq * x + \nB * (x^2) / \nC)};
		\addplot [mesh,point meta=x,line width=2pt,line join=round, line cap=round] {\nM * sin(\nRainbowStartFreq * x + \nB * (x^2) / \nC)};
	\end{axis}
\end{tikzpicture}
